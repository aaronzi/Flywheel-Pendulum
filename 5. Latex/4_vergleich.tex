\section{Vergleich beider Systeme} \label{sec:Vergleich}
In \autoref{fig:Bild2} ist die Übersicht der notwendigen Simulationsstruktur dargestellt. Aus der Übersicht geht hervor, dass beide Systeme unterschiedliche Eingänge besitzen und somit ein direkter Vergleich ohne entsprechende Berücksichtigung der Linearisierungsvorschriften unmöglich ist. Das linearisierte Modell verwendet als Eingang im Gegensatz zum nichtlinearen Modell eine Differenz $\Delta d$. Dies folgt aus \autoref{eq:Gleichung15}. Die Variable $d_{\mathrm{dyn}}$ sei gleich der Konstante $d$. Die Strukturen des nichtlinearen und des linearen Modells sind zur Information in \autoref{fig:Bild3} und \autoref{fig:Bild4} visualisiert.

\begin{figure}[H]
   \centering
   \fbox{\includegraphics[width=0.65\textwidth]{Bilder/4_vergleich/Vergleich_linear_nichtlinear_Uebersicht.pdf}}
   \caption[Übersicht der Simulationsstruktur]{Übersicht der Simulationsstruktur}
   \label{fig:Bild2}
\end{figure}
\begin{figure}[H]
   \centering
   \fbox{\includegraphics[width=0.65\textwidth]{Bilder/4_vergleich/Vergleich_linear_nichtlinear_Nichtlineare_Strecke.pdf}}
   \caption[Nichtlineare Strecke]{Nichtlineare Strecke}
   \label{fig:Bild3}
\end{figure}
\begin{figure}[H]
   \centering
   \fbox{\includegraphics[width=0.65\textwidth]{Bilder/4_vergleich/Vergleich_linear_nichtlinear_Lineare_Strecke.pdf}}
   \caption[Lineare Strecke]{Lineare Strecke}
   \label{fig:Bild4}
\end{figure}

Um das lineare mit dem nichtlinearen Modell zu vergleichen, werden gemäß \autoref{sec:Zustandsraummodell} zu den Zuständen $\Delta \underline{x}$ die Ruhelagen $\underline{x}^*$ aus \autoref{eq:Gleichung12} addiert. Aus der \autoref{fig:Bild5} und \autoref{fig:Bild6} geht hervor, dass die implementierten Systeme für kleine Abweichungen von der Ruhelagen mit steigender Zeit $"t"$ selbiges Verhalten aufweisen.

\begin{figure}[H]
   \centering
   \fbox{\includegraphics[width=0.7\textwidth]{Bilder/4_vergleich/linear_nichtlinear_vergleich_v_PV.pdf}}
   \caption[Vergleich der Spannungen $v_{\mathrm{PV}}$]{Vergleich der Spannungen $v_{\mathrm{PV}}$ bei -5V Spannungsabweichung zur Ruhelage}
   \label{fig:Bild5}
\end{figure}

\begin{figure}[H]
   \centering
   \fbox{\includegraphics[width=0.7\textwidth]{Bilder/4_vergleich/linear_nichtlinear_vergleich_i_L.pdf}}
   \caption[Vergleich der Ströme $i_{\mathrm{L}}$]{Vergleich der Ströme $i_{\mathrm{L}}$ bei -5V Spannungsabweichung zur Ruhelage}
   \label{fig:Bild6}
\end{figure}
