\pagestyle{milan}
\section{Fazit und Ausblick} \label{sec:ausblick}
Im Umfang dieser Arbeit wurde ein mathematisches Modell des Schwungradpendels, sowie der ihn treibende DC Motor, und dessen Parameter aufgestellt (\ref{sec:Modellierung}). 
Dabei wurde ein Lagrange Ansatz gewählt und eine Bewegungsgleichung hergeleitet. 
Die so entstandene Differenzialgleichung wurde anschließend in den Zustandsraum überführt (\ref{sec:Zustandsraumdarstellung}).\\

Durch eine anschließende Linearisierung des Zustandsraummodells und dessen Verifizierung (\ref{sec:Vergleich}), konnte eine gute Steuerbarkeit und Simulierbarkeit erreicht werden.\\

Die nachfolgende Sensitivitätsanalyse bestimmte die wichtigsten Parameter und deren Einfluss auf die Modellantwort. 
Dadurch konnten Erkenntnisse hinschlicht möglicher Optimierungen, sowie dem allgemeinen Verhalten des Systems gewonnen werden (\ref{sec:sesitivitaetsanalyse}).\\

Anschließend wurden eine Methode zum Aufschwingen des Pendels erarbeitet und mithilfe von Simulationen bestätigt (\ref{sec:aufschwingen}).

Der nachfolgend entwickelte Zustandsregler, der das Pendel in seiner auf geschwungenen Lage stabil hält, wurde zunächst theoretisch entwickelt und dann dessen Funktion, durch Simulationen am linearen Modell bestätigt (\ref{sec:Reglerentwurf})

Durch den nun folgenden Beobachterentwurf, wird die Implementierung der Regelung für den realen Aufbau vereinfacht, indem schwierig messbare Zustandsgrößen rekonstruiert werden.\\

Im Falle einer zukünftigen Implementierung auf einem physikalischen Modellversuch, verbleibt die in dieser Arbeit behandelten Komponente über Soft- und Hardware zu implementieren und an die realen, physikalischen Gegebenheiten des Modells anzupassen und eventuelle Optimierungen durchzuführen. 




