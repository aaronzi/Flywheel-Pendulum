\pagestyle{milan}
\section{Sensitivitätsanalyse der Modellparameter}\label{sec:sesitivitaetsanalyse}
In diesem Abschnitt wird eine Parameter- und Sensivitätsanalysise durchgeführt. 
Es wird dabei die Auswirkung von der Varianz von bestimmten Modellparametern auf die Varianz der Ausgangsparameter untersucht.

Ziel der Sensitivitätsanalyse ist es, wichtige Parameter zu identifizieren und daraus eine Optimierung der Parameter zu ermitteln.

Das Ergebniss der Sensitivitätsanalyse dient zum weiterne Verständnis des mathematischen Modelles bzw. dem zugrundeliegenden Simulationsmodell.
\subsection{Lokale und globlale Sensitivitätsanalyse}

Die verschiedenen Verfahren zur Sensitivitätsanalyse lassen sich in drei Kategorien einteilen: Lokale, globale Sensitivitätsanalyse und der sogenannenten Screening Methode.

Bei der lokalen Sensitivitätsanalyse wird für bestimmte Werte der Ausgnagsgrößen der Einfluss der Eingangsgrößen untersucht. Dabei wird immer ein Parameter variiert und die restlichen konstant gehalten (One-At-a-Time-Methode, OAT).
Die Sensitivitätsanalyse wird so für jeden Parameter einzeln durchgeführt und abschließend kann die spezifische Sensivität der einzelnen Parameter ermittelt werden.
Mathematisch entspricht dies den partiellen Ableitungen der Parameter bezüglich der Ausgangsgrößen

\subsubsection*{One-factor-at-a-time ($\pm \SI{20}{\percent}, \pm 1\, \sigma$)}

\begin{equation}
    sensitivity=\frac{\Delta Y}{\Delta X_{i}} \quad \textrm{Für jeden Parameter}\,X_i, i=1,\dots,n
\end{equation}
\begin{itemize}
    \item Nur lokale Variatizion um Arbeitspunk 
    \item Keine Korrelation zwischen Parametern
    \item Standartabweichung benötigt Annahme zur Distribution und überspannt nicht den gesammten Wertebereich
\end{itemize}

\subsubsection*{Ausdruck als Partial-Ableitung}
\begin{equation}
    sensitivity=\frac{\partial Y}{\partial X_i}
\end{equation}

%\begin{figure}   
%    % This file was created by matlab2tikz.
%
%The latest updates can be retrieved from
%  http://www.mathworks.com/matlabcentral/fileexchange/22022-matlab2tikz-matlab2tikz
%where you can also make suggestions and rate matlab2tikz.
%
\definecolor{mycolor1}{rgb}{0.00000,0.44700,0.74100}%
%
\begin{tikzpicture}

\begin{axis}[%
    %title={{$\dot\varphi\, t_{Settle}$}},
    %axis lines = left,
    xlabel = {$V_{m}\,[V]$},
    y label style={at={(axis description cs:0.3,1)},rotate=-90,anchor=south},
    ylabel = {$t_{\dot\varphi}\, [s]$},
    xmin=0,
    xmax=20,
    xtick={0,5,10,15,20},
    %ytick={0.380,0.381,0.382,0.383,0.384},
    ymajorgrids=true,
    xmajorgrids=true,
    grid style=dashed,
    y tick label style={
        /pgf/number format/.cd,
            %fixed,
           % fixed zerofill,
            precision=4,
        /tikz/.cd
    },
]
\addplot [color=mycolor1, only marks, mark size=0.5pt, mark=*, mark options={solid, black}, forget plot]
  table[row sep=crcr]{%
2.80511	0.38325\\
5.20260	0.38324\\
1.73630	0.38325\\
8.58795	0.38321\\
5.14566	0.38324\\
5.95111	0.38323\\
8.49717	0.38321\\
2.38415	0.38325\\
9.90134	0.38319\\
14.12814	0.38312\\
4.87147	0.38324\\
15.70140	0.38309\\
1.48179	0.38325\\
7.87767	0.38321\\
0.06788	0.38326\\
4.41354	0.38324\\
0.02601	0.38326\\
3.78359	0.38325\\
2.84968	0.38325\\
5.36152	0.38324\\
3.49784	0.38325\\
2.77298	0.38325\\
11.97771	0.38316\\
18.02116	0.38303\\
18.78760	0.38301\\
4.42369	0.38324\\
9.65343	0.38319\\
7.52022	0.38322\\
10.47560	0.38318\\
5.29745	0.38324\\
1.36714	0.38325\\
8.72654	0.38320\\
3.47706	0.38325\\
0.52214	0.38326\\
19.09357	0.38300\\
8.61193	0.38321\\
19.23117	0.38300\\
15.24829	0.38310\\
0.14697	0.38326\\
13.60077	0.38313\\
14.11902	0.38312\\
12.90258	0.38314\\
11.04620	0.38317\\
4.36217	0.38324\\
15.44732	0.38309\\
4.56057	0.38324\\
7.41729	0.38322\\
17.81858	0.38303\\
17.12754	0.38305\\
8.04867	0.38321\\
6.36038	0.38323\\
12.17271	0.38316\\
18.20390	0.38302\\
18.18196	0.38303\\
11.83189	0.38316\\
6.65143	0.38323\\
17.06127	0.38305\\
8.84796	0.38320\\
18.08711	0.38303\\
0.66359	0.38326\\
10.64853	0.38318\\
14.32995	0.38312\\
3.58604	0.38325\\
6.73066	0.38323\\
3.75426	0.38325\\
6.43854	0.38323\\
8.07713	0.38321\\
10.97133	0.38317\\
0.97477	0.38326\\
11.05464	0.38317\\
5.49623	0.38324\\
4.83003	0.38324\\
4.86290	0.38324\\
3.08319	0.38325\\
19.12833	0.38300\\
18.71323	0.38301\\
16.37429	0.38307\\
14.56524	0.38311\\
3.51623	0.38325\\
7.20742	0.38322\\
3.77580	0.38325\\
0.02397	0.38326\\
6.32839	0.38323\\
13.99234	0.38312\\
12.50510	0.38315\\
10.86124	0.38318\\
8.78074	0.38320\\
5.74855	0.38323\\
10.03318	0.38319\\
15.23092	0.38310\\
11.52112	0.38317\\
14.95326	0.38310\\
12.91069	0.38314\\
2.46439	0.38325\\
10.08796	0.38319\\
6.94523	0.38322\\
1.84295	0.38325\\
2.95699	0.38325\\
3.96339	0.38325\\
};
\end{axis}
\end{tikzpicture}%
%    \caption{Settling Time $\dot\varphi$}
%\end{figure}
\subsection{Parameter}
Aus den gesammten Modellparametern des Schwungradpendels (\ref{tab:Tabelle1.1}) werden folgende Parameter untersucht:
\begin{itemize}
    \item $C1,\, C2$
    \item $J1,\, J2$
    \item $m1,\, m2$
    \item $l1,\, l2$
    \item $V_m$
\end{itemize}
Es wird dabei die Auswirkung der Varrianz gennanter Parameter auf folgende Modellgrößen unterscuht:
\begin{itemize}
    \item Schwungrad Geschwindigkeit: $\dot\varphi$
    \item Schwungrad Beschleunigung: $\ddot\varphi$
    \item Pendel Winkel: $\Theta$
    \item Pendel Geschwindikeit: $\dot\Theta$
    \item Pendel Beschleunigung: $\ddot\Theta$
    \item Motor Moment: $\tau$
\end{itemize}
\subsection{Modellantwort auf Eingangssprung}
Um ein bessere Verständnis des Modelles zu erlangen, wird die Antwort wichtiger Parameter auf einen Sprung der Motorspannung $V_{\mathrm{m}}$ betratchtet.
Zugrundeliegenden ist das in\,\ref{cap:nichtlinearesZustandsraummodell} entwicklete Simulink-Modell.

%\begin{figure}   
%    % This file was created by matlab2tikz.
%
%The latest updates can be retrieved from
%  http://www.mathworks.com/matlabcentral/fileexchange/22022-matlab2tikz-matlab2tikz
%where you can also make suggestions and rate matlab2tikz.
%
\definecolor{mycolor1}{rgb}{0.00000,0.44700,0.74100}%
\definecolor{mycolor2}{rgb}{0.85000,0.32500,0.09800}%
\definecolor{mycolor3}{rgb}{0.92900,0.69400,0.12500}%
\definecolor{mycolor4}{rgb}{0.49400,0.18400,0.55600}%
\definecolor{mycolor5}{rgb}{0.46600,0.67400,0.18800}%
\definecolor{mycolor6}{rgb}{0.30100,0.74500,0.93300}%
\definecolor{mycolor7}{rgb}{0.63500,0.07800,0.18400}%
%
\begin{tikzpicture}

\begin{axis}[%
%width=5.95in,
%height=4.975in,
xlabel = {$V_{m}\,[V]$},
    y label style={at={(axis description cs:0.3,1)},rotate=-90,anchor=south},
    ylabel = {$t_{\dot\varphi}\, [s]$},
%at={(0.998in,0.671in)},
scale only axis,
xmin=0.0000,
xmax=10.0000,
ymin=0.0000,
ymax=250.0000,
axis background/.style={fill=white},
xmajorgrids,
ymajorgrids
]
\addplot [color=mycolor1, forget plot]
  table[row sep=crcr]{%
0.0000	0.0000\\
0.0203	10.2440\\
0.0406	18.3763\\
0.0609	24.8291\\
0.0811	29.9244\\
0.1013	33.9678\\
0.1215	37.1744\\
0.1417	39.7160\\
0.1618	41.7204\\
0.1819	43.3094\\
0.2020	44.5686\\
0.2221	45.5668\\
0.2423	46.3620\\
0.2626	46.9956\\
0.2830	47.5013\\
0.3036	47.9078\\
0.3246	48.2386\\
0.3460	48.5090\\
0.3681	48.7344\\
0.3912	48.9263\\
0.4157	49.0937\\
0.4421	49.2443\\
0.4711	49.3843\\
0.5033	49.5182\\
0.5385	49.6459\\
0.5750	49.7621\\
0.6101	49.8596\\
0.6423	49.9365\\
0.6715	49.9951\\
0.6979	50.0385\\
0.7219	50.0695\\
0.7439	50.0908\\
0.7644	50.1044\\
0.7839	50.1117\\
0.8029	50.1137\\
0.8220	50.1109\\
0.8420	50.1028\\
0.8636	50.0887\\
0.8875	50.0672\\
0.9150	50.0360\\
0.9483	49.9908\\
0.9959	49.9176\\
1.0762	49.7941\\
1.1123	49.7472\\
1.1424	49.7152\\
1.1690	49.6933\\
1.1934	49.6790\\
1.2164	49.6708\\
1.2387	49.6679\\
1.2610	49.6698\\
1.2840	49.6767\\
1.3085	49.6891\\
1.3355	49.7083\\
1.3665	49.7363\\
1.4050	49.7775\\
1.4773	49.8634\\
1.5292	49.9214\\
1.5657	49.9557\\
1.5969	49.9790\\
1.6251	49.9945\\
1.6515	50.0038\\
1.6771	50.0079\\
1.7025	50.0071\\
1.7286	50.0016\\
1.7563	49.9908\\
1.7868	49.9737\\
1.8223	49.9483\\
1.8691	49.9088\\
1.9802	49.8126\\
2.0185	49.7866\\
2.0518	49.7693\\
2.0825	49.7585\\
2.1118	49.7530\\
2.1408	49.7523\\
2.1705	49.7563\\
2.2019	49.7654\\
2.2365	49.7803\\
2.2775	49.8033\\
2.3375	49.8428\\
2.4185	49.8955\\
2.4618	49.9181\\
2.4990	49.9326\\
2.5335	49.9411\\
2.5669	49.9447\\
2.6004	49.9436\\
2.6353	49.9377\\
2.6733	49.9265\\
2.7178	49.9083\\
2.7825	49.8764\\
2.8691	49.8344\\
2.9161	49.8167\\
2.9571	49.8060\\
2.9957	49.8006\\
3.0339	49.8000\\
3.0733	49.8040\\
3.1160	49.8130\\
3.1664	49.8285\\
3.2488	49.8595\\
3.3249	49.8863\\
3.3758	49.8995\\
3.4212	49.9067\\
3.4650	49.9090\\
3.5096	49.9067\\
3.5578	49.8995\\
3.6152	49.8862\\
3.8211	49.8337\\
3.8729	49.8279\\
3.9232	49.8269\\
3.9754	49.8303\\
4.0347	49.8390\\
4.1205	49.8567\\
4.2284	49.8781\\
4.2916	49.8859\\
4.3496	49.8886\\
4.4085	49.8868\\
4.4746	49.8800\\
4.5722	49.8649\\
4.6847	49.8486\\
4.7552	49.8430\\
4.8216	49.8423\\
4.8925	49.8461\\
4.9847	49.8559\\
5.1462	49.8734\\
5.2249	49.8770\\
5.3023	49.8759\\
5.3934	49.8699\\
5.6347	49.8515\\
5.7228	49.8510\\
5.8212	49.8551\\
6.1149	49.8705\\
6.2186	49.8691\\
6.3694	49.8620\\
6.5256	49.8561\\
6.6429	49.8562\\
6.7998	49.8615\\
6.9860	49.8667\\
7.1232	49.8658\\
7.5490	49.8590\\
8.0549	49.8636\\
8.4238	49.8602\\
9.0433	49.8620\\
9.3586	49.8612\\
9.9161	49.8621\\
10.0000	49.8617\\
};
\addplot [color=mycolor2, forget plot]
  table[row sep=crcr]{%
0.0000	0.0000\\
0.0203	20.4881\\
0.0406	36.7526\\
0.0609	49.6583\\
0.0811	59.8489\\
0.1013	67.9355\\
0.1215	74.3489\\
0.1417	79.4321\\
0.1618	83.4411\\
0.1819	86.6192\\
0.2020	89.1380\\
0.2221	91.1345\\
0.2423	92.7252\\
0.2625	93.9872\\
0.2829	94.9999\\
0.3035	95.8138\\
0.3244	96.4733\\
0.3457	97.0127\\
0.3677	97.4627\\
0.3906	97.8442\\
0.4148	98.1763\\
0.4409	98.4752\\
0.4695	98.7528\\
0.5011	99.0172\\
0.5357	99.2701\\
0.5716	99.5010\\
0.6062	99.6960\\
0.6379	99.8507\\
0.6664	99.9688\\
0.6919	100.0566\\
0.7148	100.1203\\
0.7353	100.1648\\
0.7537	100.1946\\
0.7703	100.2132\\
0.7854	100.2233\\
0.7995	100.2270\\
0.8133	100.2254\\
0.8276	100.2185\\
0.8431	100.2051\\
0.8603	100.1836\\
0.8799	100.1510\\
0.9025	100.1041\\
0.9293	100.0375\\
0.9632	99.9405\\
1.0213	99.7580\\
1.0754	99.5937\\
1.1093	99.5044\\
1.1372	99.4428\\
1.1613	99.3999\\
1.1826	99.3708\\
1.2018	99.3519\\
1.2194	99.3410\\
1.2359	99.3364\\
1.2520	99.3369\\
1.2684	99.3426\\
1.2858	99.3541\\
1.3048	99.3726\\
1.3261	99.4003\\
1.3505	99.4400\\
1.3795	99.4964\\
1.4171	99.5800\\
1.5411	99.8640\\
1.5718	99.9192\\
1.5981	99.9577\\
1.6214	99.9842\\
1.6425	100.0014\\
1.6621	100.0116\\
1.6807	100.0159\\
1.6990	100.0152\\
1.7177	100.0095\\
1.7375	99.9982\\
1.7591	99.9802\\
1.7833	99.9536\\
1.8113	99.9156\\
1.8459	99.8605\\
1.8996	99.7652\\
1.9644	99.6523\\
2.0004	99.5984\\
2.0301	99.5616\\
2.0563	99.5361\\
2.0800	99.5193\\
2.1021	99.5091\\
2.1233	99.5045\\
2.1442	99.5049\\
2.1656	99.5102\\
2.1881	99.5208\\
2.2126	99.5378\\
2.2401	99.5628\\
2.2726	99.5988\\
2.3157	99.6540\\
2.4228	99.7941\\
2.4579	99.8311\\
2.4879	99.8565\\
2.5150	99.8735\\
2.5402	99.8840\\
2.5644	99.8890\\
2.5883	99.8891\\
2.6127	99.8845\\
2.6384	99.8746\\
2.6664	99.8587\\
2.6982	99.8350\\
2.7372	99.7999\\
2.8036	99.7328\\
2.8625	99.6759\\
2.9011	99.6447\\
2.9339	99.6238\\
2.9637	99.6101\\
2.9918	99.6023\\
3.0192	99.5994\\
3.0467	99.6013\\
3.0753	99.6081\\
3.1061	99.6202\\
3.1408	99.6392\\
3.1833	99.6679\\
3.2680	99.7323\\
3.3210	99.7692\\
3.3606	99.7914\\
3.3954	99.8058\\
3.4278	99.8143\\
3.4592	99.8178\\
3.4907	99.8166\\
3.5234	99.8106\\
3.5589	99.7993\\
3.5997	99.7812\\
3.6540	99.7518\\
3.7645	99.6907\\
3.8086	99.6722\\
3.8474	99.6608\\
3.8839	99.6548\\
3.9198	99.6535\\
3.9566	99.6569\\
3.9960	99.6653\\
4.0409	99.6797\\
4.1012	99.7042\\
4.2139	99.7508\\
4.2625	99.7654\\
4.3057	99.7738\\
4.3470	99.7771\\
4.3886	99.7759\\
4.4326	99.7699\\
4.4826	99.7583\\
4.5507	99.7375\\
4.6640	99.7026\\
4.7181	99.6910\\
4.7666	99.6853\\
4.8139	99.6842\\
4.8629	99.6877\\
4.9178	99.6964\\
4.9911	99.7129\\
5.1193	99.7422\\
5.1791	99.7508\\
5.2338	99.7541\\
5.2888	99.7528\\
5.3489	99.7468\\
5.4266	99.7342\\
5.5790	99.7085\\
5.6451	99.7026\\
5.7076	99.7015\\
5.7734	99.7050\\
5.8543	99.7141\\
6.0488	99.7379\\
6.1220	99.7411\\
6.1951	99.7397\\
6.2806	99.7334\\
6.5265	99.7122\\
6.6093	99.7115\\
6.7001	99.7154\\
7.0221	99.7337\\
7.1213	99.7317\\
7.2766	99.7231\\
7.4155	99.7173\\
7.5255	99.7173\\
7.6632	99.7222\\
7.8693	99.7293\\
7.9970	99.7287\\
8.2189	99.7219\\
8.3740	99.7200\\
8.5393	99.7227\\
8.8062	99.7272\\
8.9952	99.7249\\
9.2738	99.7218\\
9.5302	99.7247\\
9.7712	99.7256\\
10.0000	99.7234\\
};
\addplot [color=mycolor3, forget plot]
  table[row sep=crcr]{%
0.0000	0.0000\\
0.0203	30.7321\\
0.0406	55.1288\\
0.0609	74.4875\\
0.0811	89.7733\\
0.1013	101.9034\\
0.1215	111.5235\\
0.1417	119.1485\\
0.1618	125.1624\\
0.1819	129.9299\\
0.2020	133.7087\\
0.2221	136.7042\\
0.2423	139.0908\\
0.2625	140.9845\\
0.2829	142.5039\\
0.3035	143.7251\\
0.3244	144.7142\\
0.3457	145.5228\\
0.3676	146.1942\\
0.3904	146.7634\\
0.4145	147.2588\\
0.4405	147.7047\\
0.4689	148.1175\\
0.5004	148.5122\\
0.5350	148.8914\\
0.5710	149.2389\\
0.6057	149.5330\\
0.6374	149.7658\\
0.6658	149.9434\\
0.6912	150.0758\\
0.7139	150.1719\\
0.7342	150.2397\\
0.7522	150.2851\\
0.7681	150.3138\\
0.7822	150.3303\\
0.7947	150.3380\\
0.8064	150.3395\\
0.8180	150.3355\\
0.8304	150.3255\\
0.8443	150.3074\\
0.8601	150.2785\\
0.8783	150.2349\\
0.8994	150.1719\\
0.9244	150.0823\\
0.9554	149.9535\\
1.0004	149.7453\\
1.0756	149.3978\\
1.1092	149.2640\\
1.1367	149.1717\\
1.1603	149.1073\\
1.1810	149.0631\\
1.1992	149.0342\\
1.2154	149.0167\\
1.2300	149.0075\\
1.2436	149.0046\\
1.2570	149.0070\\
1.2709	149.0149\\
1.2860	149.0294\\
1.3029	149.0528\\
1.3220	149.0877\\
1.3440	149.1379\\
1.3700	149.2092\\
1.4022	149.3113\\
1.4505	149.4813\\
1.5195	149.7222\\
1.5538	149.8258\\
1.5818	149.8970\\
1.6059	149.9468\\
1.6272	149.9810\\
1.6462	150.0032\\
1.6634	150.0165\\
1.6793	150.0229\\
1.6946	150.0237\\
1.7099	150.0195\\
1.7260	150.0097\\
1.7436	149.9931\\
1.7632	149.9676\\
1.7855	149.9306\\
1.8116	149.8779\\
1.8436	149.8026\\
1.8899	149.6808\\
1.9665	149.4793\\
2.0011	149.4012\\
2.0295	149.3477\\
2.0541	149.3104\\
2.0760	149.2851\\
2.0959	149.2689\\
2.1143	149.2598\\
2.1318	149.2564\\
2.1491	149.2580\\
2.1670	149.2648\\
2.1860	149.2774\\
2.2068	149.2972\\
2.2302	149.3262\\
2.2574	149.3676\\
2.2909	149.4274\\
2.3409	149.5270\\
2.4112	149.6659\\
2.4468	149.7263\\
2.4761	149.7676\\
2.5017	149.7962\\
2.5248	149.8155\\
2.5461	149.8273\\
2.5664	149.8333\\
2.5862	149.8341\\
2.6062	149.8301\\
2.6271	149.8207\\
2.6497	149.8053\\
2.6748	149.7821\\
2.7038	149.7488\\
2.7395	149.7003\\
2.7968	149.6137\\
2.8585	149.5228\\
2.8950	149.4768\\
2.9254	149.4454\\
2.9524	149.4238\\
2.9772	149.4098\\
3.0005	149.4019\\
3.0232	149.3992\\
3.0459	149.4013\\
3.0694	149.4084\\
3.0945	149.4210\\
3.1222	149.4405\\
3.1541	149.4688\\
3.1944	149.5111\\
3.3329	149.6623\\
3.3665	149.6894\\
3.3962	149.7077\\
3.4237	149.7194\\
3.4499	149.7255\\
3.4757	149.7267\\
3.5019	149.7232\\
3.5294	149.7146\\
3.5592	149.7003\\
3.5931	149.6786\\
3.6352	149.6459\\
3.7858	149.5234\\
3.8214	149.5032\\
3.8535	149.4901\\
3.8838	149.4827\\
3.9134	149.4802\\
3.9433	149.4824\\
3.9745	149.4895\\
4.0084	149.5020\\
4.0474	149.5216\\
4.0988	149.5530\\
4.2120	149.6237\\
4.2542	149.6438\\
4.2912	149.6565\\
4.3258	149.6636\\
4.3597	149.6658\\
4.3940	149.6634\\
4.4302	149.6562\\
4.4704	149.6433\\
4.5195	149.6225\\
4.6900	149.5455\\
4.7326	149.5338\\
4.7722	149.5276\\
4.8110	149.5261\\
4.8508	149.5293\\
4.8936	149.5374\\
4.9435	149.5517\\
5.0183	149.5784\\
5.1059	149.6087\\
5.1581	149.6219\\
5.2044	149.6290\\
5.2490	149.6313\\
5.2944	149.6290\\
5.3436	149.6218\\
5.4029	149.6084\\
5.5990	149.5597\\
5.6522	149.5536\\
5.7034	149.5522\\
5.7564	149.5553\\
5.8163	149.5636\\
5.9022	149.5805\\
6.0133	149.6018\\
6.0774	149.6093\\
6.1364	149.6117\\
6.1966	149.6096\\
6.2651	149.6025\\
6.3760	149.5857\\
6.4756	149.5724\\
6.5465	149.5675\\
6.6143	149.5673\\
6.6880	149.5717\\
6.7914	149.5829\\
6.9262	149.5968\\
7.0067	149.6004\\
7.0855	149.5994\\
7.1784	149.5936\\
7.4181	149.5760\\
7.5079	149.5755\\
7.6088	149.5796\\
7.8942	149.5942\\
7.9995	149.5930\\
8.1507	149.5862\\
8.3109	149.5803\\
8.4306	149.5806\\
8.5965	149.5861\\
8.7752	149.5906\\
8.9159	149.5895\\
9.3181	149.5830\\
9.8805	149.5869\\
10.0000	149.5851\\
};
\addplot [color=mycolor4, forget plot]
  table[row sep=crcr]{%
0.0000	0.0000\\
0.0203	10.2440\\
0.0406	18.3763\\
0.0609	24.8291\\
0.0811	29.9244\\
0.1013	33.9678\\
0.1215	37.1744\\
0.1417	39.7160\\
0.1618	41.7204\\
0.1819	43.3094\\
0.2020	44.5686\\
0.2221	45.5668\\
0.2423	46.3620\\
0.2626	46.9956\\
0.2830	47.5013\\
0.3036	47.9078\\
0.3246	48.2386\\
0.3460	48.5090\\
0.3681	48.7344\\
0.3912	48.9263\\
0.4157	49.0937\\
0.4421	49.2443\\
0.4711	49.3843\\
0.5033	49.5182\\
0.5385	49.6459\\
0.5750	49.7621\\
0.6101	49.8596\\
0.6423	49.9365\\
0.6715	49.9951\\
0.6979	50.0385\\
0.7219	50.0695\\
0.7439	50.0908\\
0.7644	50.1044\\
0.7839	50.1117\\
0.8029	50.1137\\
0.8220	50.1109\\
0.8420	50.1028\\
0.8636	50.0887\\
0.8875	50.0672\\
0.9150	50.0360\\
0.9483	49.9908\\
0.9959	49.9176\\
1.0762	49.7941\\
1.1123	49.7472\\
1.1424	49.7152\\
1.1690	49.6933\\
1.1934	49.6790\\
1.2164	49.6708\\
1.2387	49.6679\\
1.2610	49.6698\\
1.2840	49.6767\\
1.3085	49.6891\\
1.3355	49.7083\\
1.3665	49.7363\\
1.4050	49.7775\\
1.4773	49.8634\\
1.5292	49.9214\\
1.5657	49.9557\\
1.5969	49.9790\\
1.6251	49.9945\\
1.6515	50.0038\\
1.6771	50.0079\\
1.7025	50.0071\\
1.7286	50.0016\\
1.7563	49.9908\\
1.7868	49.9737\\
1.8223	49.9483\\
1.8691	49.9088\\
1.9802	49.8126\\
2.0185	49.7866\\
2.0518	49.7693\\
2.0825	49.7585\\
2.1118	49.7530\\
2.1408	49.7523\\
2.1705	49.7563\\
2.2019	49.7654\\
2.2365	49.7803\\
2.2775	49.8033\\
2.3375	49.8428\\
2.4185	49.8955\\
2.4618	49.9181\\
2.4990	49.9326\\
2.5335	49.9411\\
2.5669	49.9447\\
2.6004	49.9436\\
2.6353	49.9377\\
2.6733	49.9265\\
2.7178	49.9083\\
2.7825	49.8764\\
2.8691	49.8344\\
2.9161	49.8167\\
2.9571	49.8060\\
2.9957	49.8006\\
3.0339	49.8000\\
3.0733	49.8040\\
3.1160	49.8130\\
3.1664	49.8285\\
3.2488	49.8595\\
3.3249	49.8863\\
3.3758	49.8995\\
3.4212	49.9067\\
3.4650	49.9090\\
3.5096	49.9067\\
3.5578	49.8995\\
3.6152	49.8862\\
3.8211	49.8337\\
3.8729	49.8279\\
3.9232	49.8269\\
3.9754	49.8303\\
4.0347	49.8390\\
4.1205	49.8567\\
4.2284	49.8781\\
4.2916	49.8859\\
4.3496	49.8886\\
4.4085	49.8868\\
4.4746	49.8800\\
4.5722	49.8649\\
4.6847	49.8486\\
4.7552	49.8430\\
4.8216	49.8423\\
4.8925	49.8461\\
4.9847	49.8559\\
5.1462	49.8734\\
5.2249	49.8770\\
5.3023	49.8759\\
5.3934	49.8699\\
5.6347	49.8515\\
5.7228	49.8510\\
5.8212	49.8551\\
6.1149	49.8705\\
6.2186	49.8691\\
6.3694	49.8620\\
6.5256	49.8561\\
6.6429	49.8562\\
6.7998	49.8615\\
6.9860	49.8667\\
7.1232	49.8658\\
7.5490	49.8590\\
8.0549	49.8636\\
8.4238	49.8602\\
9.0433	49.8620\\
9.3586	49.8612\\
9.9161	49.8621\\
10.0000	49.8617\\
};
\addplot [color=mycolor5, forget plot]
  table[row sep=crcr]{%
0.0000	0.0000\\
0.0203	20.4881\\
0.0406	36.7526\\
0.0609	49.6583\\
0.0811	59.8489\\
0.1013	67.9355\\
0.1215	74.3489\\
0.1417	79.4321\\
0.1618	83.4411\\
0.1819	86.6192\\
0.2020	89.1380\\
0.2221	91.1345\\
0.2423	92.7252\\
0.2625	93.9872\\
0.2829	94.9999\\
0.3035	95.8138\\
0.3244	96.4733\\
0.3457	97.0127\\
0.3677	97.4627\\
0.3906	97.8442\\
0.4148	98.1763\\
0.4409	98.4752\\
0.4695	98.7528\\
0.5011	99.0172\\
0.5357	99.2701\\
0.5716	99.5010\\
0.6062	99.6960\\
0.6379	99.8507\\
0.6664	99.9688\\
0.6919	100.0566\\
0.7148	100.1203\\
0.7353	100.1648\\
0.7537	100.1946\\
0.7703	100.2132\\
0.7854	100.2233\\
0.7995	100.2270\\
0.8133	100.2254\\
0.8276	100.2185\\
0.8431	100.2051\\
0.8603	100.1836\\
0.8799	100.1510\\
0.9025	100.1041\\
0.9293	100.0375\\
0.9632	99.9405\\
1.0213	99.7580\\
1.0754	99.5937\\
1.1093	99.5044\\
1.1372	99.4428\\
1.1613	99.3999\\
1.1826	99.3708\\
1.2018	99.3519\\
1.2194	99.3410\\
1.2359	99.3364\\
1.2520	99.3369\\
1.2684	99.3426\\
1.2858	99.3541\\
1.3048	99.3726\\
1.3261	99.4003\\
1.3505	99.4400\\
1.3795	99.4964\\
1.4171	99.5800\\
1.5411	99.8640\\
1.5718	99.9192\\
1.5981	99.9577\\
1.6214	99.9842\\
1.6425	100.0014\\
1.6621	100.0116\\
1.6807	100.0159\\
1.6990	100.0152\\
1.7177	100.0095\\
1.7375	99.9982\\
1.7591	99.9802\\
1.7833	99.9536\\
1.8113	99.9156\\
1.8459	99.8605\\
1.8996	99.7652\\
1.9644	99.6523\\
2.0004	99.5984\\
2.0301	99.5616\\
2.0563	99.5361\\
2.0800	99.5193\\
2.1021	99.5091\\
2.1233	99.5045\\
2.1442	99.5049\\
2.1656	99.5102\\
2.1881	99.5208\\
2.2126	99.5378\\
2.2401	99.5628\\
2.2726	99.5988\\
2.3157	99.6540\\
2.4228	99.7941\\
2.4579	99.8311\\
2.4879	99.8565\\
2.5150	99.8735\\
2.5402	99.8840\\
2.5644	99.8890\\
2.5883	99.8891\\
2.6127	99.8845\\
2.6384	99.8746\\
2.6664	99.8587\\
2.6982	99.8350\\
2.7372	99.7999\\
2.8036	99.7328\\
2.8625	99.6759\\
2.9011	99.6447\\
2.9339	99.6238\\
2.9637	99.6101\\
2.9918	99.6023\\
3.0192	99.5994\\
3.0467	99.6013\\
3.0753	99.6081\\
3.1061	99.6202\\
3.1408	99.6392\\
3.1833	99.6679\\
3.2680	99.7323\\
3.3210	99.7692\\
3.3606	99.7914\\
3.3954	99.8058\\
3.4278	99.8143\\
3.4592	99.8178\\
3.4907	99.8166\\
3.5234	99.8106\\
3.5589	99.7993\\
3.5997	99.7812\\
3.6540	99.7518\\
3.7645	99.6907\\
3.8086	99.6722\\
3.8474	99.6608\\
3.8839	99.6548\\
3.9198	99.6535\\
3.9566	99.6569\\
3.9960	99.6653\\
4.0409	99.6797\\
4.1012	99.7042\\
4.2139	99.7508\\
4.2625	99.7654\\
4.3057	99.7738\\
4.3470	99.7771\\
4.3886	99.7759\\
4.4326	99.7699\\
4.4826	99.7583\\
4.5507	99.7375\\
4.6640	99.7026\\
4.7181	99.6910\\
4.7666	99.6853\\
4.8139	99.6842\\
4.8629	99.6877\\
4.9178	99.6964\\
4.9911	99.7129\\
5.1193	99.7422\\
5.1791	99.7508\\
5.2338	99.7541\\
5.2888	99.7528\\
5.3489	99.7468\\
5.4266	99.7342\\
5.5790	99.7085\\
5.6451	99.7026\\
5.7076	99.7015\\
5.7734	99.7050\\
5.8543	99.7141\\
6.0488	99.7379\\
6.1220	99.7411\\
6.1951	99.7397\\
6.2806	99.7334\\
6.5265	99.7122\\
6.6093	99.7115\\
6.7001	99.7154\\
7.0221	99.7337\\
7.1213	99.7317\\
7.2766	99.7231\\
7.4155	99.7173\\
7.5255	99.7173\\
7.6632	99.7222\\
7.8693	99.7293\\
7.9970	99.7287\\
8.2189	99.7219\\
8.3740	99.7200\\
8.5393	99.7227\\
8.8062	99.7272\\
8.9952	99.7249\\
9.2738	99.7218\\
9.5302	99.7247\\
9.7712	99.7256\\
10.0000	99.7234\\
};
\addplot [color=mycolor6, forget plot]
  table[row sep=crcr]{%
0.0000	0.0000\\
0.0203	30.7321\\
0.0406	55.1288\\
0.0609	74.4875\\
0.0811	89.7733\\
0.1013	101.9034\\
0.1215	111.5235\\
0.1417	119.1485\\
0.1618	125.1624\\
0.1819	129.9299\\
0.2020	133.7087\\
0.2221	136.7042\\
0.2423	139.0908\\
0.2625	140.9845\\
0.2829	142.5039\\
0.3035	143.7251\\
0.3244	144.7142\\
0.3457	145.5228\\
0.3676	146.1942\\
0.3904	146.7634\\
0.4145	147.2588\\
0.4405	147.7047\\
0.4689	148.1175\\
0.5004	148.5122\\
0.5350	148.8914\\
0.5710	149.2389\\
0.6057	149.5330\\
0.6374	149.7658\\
0.6658	149.9434\\
0.6912	150.0758\\
0.7139	150.1719\\
0.7342	150.2397\\
0.7522	150.2851\\
0.7681	150.3138\\
0.7822	150.3303\\
0.7947	150.3380\\
0.8064	150.3395\\
0.8180	150.3355\\
0.8304	150.3255\\
0.8443	150.3074\\
0.8601	150.2785\\
0.8783	150.2349\\
0.8994	150.1719\\
0.9244	150.0823\\
0.9554	149.9535\\
1.0004	149.7453\\
1.0756	149.3978\\
1.1092	149.2640\\
1.1367	149.1717\\
1.1603	149.1073\\
1.1810	149.0631\\
1.1992	149.0342\\
1.2154	149.0167\\
1.2300	149.0075\\
1.2436	149.0046\\
1.2570	149.0070\\
1.2709	149.0149\\
1.2860	149.0294\\
1.3029	149.0528\\
1.3220	149.0877\\
1.3440	149.1379\\
1.3700	149.2092\\
1.4022	149.3113\\
1.4505	149.4813\\
1.5195	149.7222\\
1.5538	149.8258\\
1.5818	149.8970\\
1.6059	149.9468\\
1.6272	149.9810\\
1.6462	150.0032\\
1.6634	150.0165\\
1.6793	150.0229\\
1.6946	150.0237\\
1.7099	150.0195\\
1.7260	150.0097\\
1.7436	149.9931\\
1.7632	149.9676\\
1.7855	149.9306\\
1.8116	149.8779\\
1.8436	149.8026\\
1.8899	149.6808\\
1.9665	149.4793\\
2.0011	149.4012\\
2.0295	149.3477\\
2.0541	149.3104\\
2.0760	149.2851\\
2.0959	149.2689\\
2.1143	149.2598\\
2.1318	149.2564\\
2.1491	149.2580\\
2.1670	149.2648\\
2.1860	149.2774\\
2.2068	149.2972\\
2.2302	149.3262\\
2.2574	149.3676\\
2.2909	149.4274\\
2.3409	149.5270\\
2.4112	149.6659\\
2.4468	149.7263\\
2.4761	149.7676\\
2.5017	149.7962\\
2.5248	149.8155\\
2.5461	149.8273\\
2.5664	149.8333\\
2.5862	149.8341\\
2.6062	149.8301\\
2.6271	149.8207\\
2.6497	149.8053\\
2.6748	149.7821\\
2.7038	149.7488\\
2.7395	149.7003\\
2.7968	149.6137\\
2.8585	149.5228\\
2.8950	149.4768\\
2.9254	149.4454\\
2.9524	149.4238\\
2.9772	149.4098\\
3.0005	149.4019\\
3.0232	149.3992\\
3.0459	149.4013\\
3.0694	149.4084\\
3.0945	149.4210\\
3.1222	149.4405\\
3.1541	149.4688\\
3.1944	149.5111\\
3.3329	149.6623\\
3.3665	149.6894\\
3.3962	149.7077\\
3.4237	149.7194\\
3.4499	149.7255\\
3.4757	149.7267\\
3.5019	149.7232\\
3.5294	149.7146\\
3.5592	149.7003\\
3.5931	149.6786\\
3.6352	149.6459\\
3.7858	149.5234\\
3.8214	149.5032\\
3.8535	149.4901\\
3.8838	149.4827\\
3.9134	149.4802\\
3.9433	149.4824\\
3.9745	149.4895\\
4.0084	149.5020\\
4.0474	149.5216\\
4.0988	149.5530\\
4.2120	149.6237\\
4.2542	149.6438\\
4.2912	149.6565\\
4.3258	149.6636\\
4.3597	149.6658\\
4.3940	149.6634\\
4.4302	149.6562\\
4.4704	149.6433\\
4.5195	149.6225\\
4.6900	149.5455\\
4.7326	149.5338\\
4.7722	149.5276\\
4.8110	149.5261\\
4.8508	149.5293\\
4.8936	149.5374\\
4.9435	149.5517\\
5.0183	149.5784\\
5.1059	149.6087\\
5.1581	149.6219\\
5.2044	149.6290\\
5.2490	149.6313\\
5.2944	149.6290\\
5.3436	149.6218\\
5.4029	149.6084\\
5.5990	149.5597\\
5.6522	149.5536\\
5.7034	149.5522\\
5.7564	149.5553\\
5.8163	149.5636\\
5.9022	149.5805\\
6.0133	149.6018\\
6.0774	149.6093\\
6.1364	149.6117\\
6.1966	149.6096\\
6.2651	149.6025\\
6.3760	149.5857\\
6.4756	149.5724\\
6.5465	149.5675\\
6.6143	149.5673\\
6.6880	149.5717\\
6.7914	149.5829\\
6.9262	149.5968\\
7.0067	149.6004\\
7.0855	149.5994\\
7.1784	149.5936\\
7.4181	149.5760\\
7.5079	149.5755\\
7.6088	149.5796\\
7.8942	149.5942\\
7.9995	149.5930\\
8.1507	149.5862\\
8.3109	149.5803\\
8.4306	149.5806\\
8.5965	149.5861\\
8.7752	149.5906\\
8.9159	149.5895\\
9.3181	149.5830\\
9.8805	149.5869\\
10.0000	149.5851\\
};
\addplot [color=mycolor7, forget plot]
  table[row sep=crcr]{%
0.0000	0.0000\\
0.0203	10.2440\\
0.0406	18.3763\\
0.0609	24.8291\\
0.0811	29.9244\\
0.1013	33.9678\\
0.1215	37.1744\\
0.1417	39.7160\\
0.1618	41.7204\\
0.1819	43.3094\\
0.2020	44.5686\\
0.2221	45.5668\\
0.2423	46.3620\\
0.2626	46.9956\\
0.2830	47.5013\\
0.3036	47.9078\\
0.3246	48.2386\\
0.3460	48.5090\\
0.3681	48.7344\\
0.3912	48.9263\\
0.4157	49.0937\\
0.4421	49.2443\\
0.4711	49.3843\\
0.5033	49.5182\\
0.5385	49.6459\\
0.5750	49.7621\\
0.6101	49.8596\\
0.6423	49.9365\\
0.6715	49.9951\\
0.6979	50.0385\\
0.7219	50.0695\\
0.7439	50.0908\\
0.7644	50.1044\\
0.7839	50.1117\\
0.8029	50.1137\\
0.8220	50.1109\\
0.8420	50.1028\\
0.8636	50.0887\\
0.8875	50.0672\\
0.9150	50.0360\\
0.9483	49.9908\\
0.9959	49.9176\\
1.0762	49.7941\\
1.1123	49.7472\\
1.1424	49.7152\\
1.1690	49.6933\\
1.1934	49.6790\\
1.2164	49.6708\\
1.2387	49.6679\\
1.2610	49.6698\\
1.2840	49.6767\\
1.3085	49.6891\\
1.3355	49.7083\\
1.3665	49.7363\\
1.4050	49.7775\\
1.4773	49.8634\\
1.5292	49.9214\\
1.5657	49.9557\\
1.5969	49.9790\\
1.6251	49.9945\\
1.6515	50.0038\\
1.6771	50.0079\\
1.7025	50.0071\\
1.7286	50.0016\\
1.7563	49.9908\\
1.7868	49.9737\\
1.8223	49.9483\\
1.8691	49.9088\\
1.9802	49.8126\\
2.0185	49.7866\\
2.0518	49.7693\\
2.0825	49.7585\\
2.1118	49.7530\\
2.1408	49.7523\\
2.1705	49.7563\\
2.2019	49.7654\\
2.2365	49.7803\\
2.2775	49.8033\\
2.3375	49.8428\\
2.4185	49.8955\\
2.4618	49.9181\\
2.4990	49.9326\\
2.5335	49.9411\\
2.5669	49.9447\\
2.6004	49.9436\\
2.6353	49.9377\\
2.6733	49.9265\\
2.7178	49.9083\\
2.7825	49.8764\\
2.8691	49.8344\\
2.9161	49.8167\\
2.9571	49.8060\\
2.9957	49.8006\\
3.0339	49.8000\\
3.0733	49.8040\\
3.1160	49.8130\\
3.1664	49.8285\\
3.2488	49.8595\\
3.3249	49.8863\\
3.3758	49.8995\\
3.4212	49.9067\\
3.4650	49.9090\\
3.5096	49.9067\\
3.5578	49.8995\\
3.6152	49.8862\\
3.8211	49.8337\\
3.8729	49.8279\\
3.9232	49.8269\\
3.9754	49.8303\\
4.0347	49.8390\\
4.1205	49.8567\\
4.2284	49.8781\\
4.2916	49.8859\\
4.3496	49.8886\\
4.4085	49.8868\\
4.4746	49.8800\\
4.5722	49.8649\\
4.6847	49.8486\\
4.7552	49.8430\\
4.8216	49.8423\\
4.8925	49.8461\\
4.9847	49.8559\\
5.1462	49.8734\\
5.2249	49.8770\\
5.3023	49.8759\\
5.3934	49.8699\\
5.6347	49.8515\\
5.7228	49.8510\\
5.8212	49.8551\\
6.1149	49.8705\\
6.2186	49.8691\\
6.3694	49.8620\\
6.5256	49.8561\\
6.6429	49.8562\\
6.7998	49.8615\\
6.9860	49.8667\\
7.1232	49.8658\\
7.5490	49.8590\\
8.0549	49.8636\\
8.4238	49.8602\\
9.0433	49.8620\\
9.3586	49.8612\\
9.9161	49.8621\\
10.0000	49.8617\\
};
\addplot [color=mycolor1, forget plot]
  table[row sep=crcr]{%
0.0000	0.0000\\
0.0203	20.4881\\
0.0406	36.7526\\
0.0609	49.6583\\
0.0811	59.8489\\
0.1013	67.9355\\
0.1215	74.3489\\
0.1417	79.4321\\
0.1618	83.4411\\
0.1819	86.6192\\
0.2020	89.1380\\
0.2221	91.1345\\
0.2423	92.7252\\
0.2625	93.9872\\
0.2829	94.9999\\
0.3035	95.8138\\
0.3244	96.4733\\
0.3457	97.0127\\
0.3677	97.4627\\
0.3906	97.8442\\
0.4148	98.1763\\
0.4409	98.4752\\
0.4695	98.7528\\
0.5011	99.0172\\
0.5357	99.2701\\
0.5716	99.5010\\
0.6062	99.6960\\
0.6379	99.8507\\
0.6664	99.9688\\
0.6919	100.0566\\
0.7148	100.1203\\
0.7353	100.1648\\
0.7537	100.1946\\
0.7703	100.2132\\
0.7854	100.2233\\
0.7995	100.2270\\
0.8133	100.2254\\
0.8276	100.2185\\
0.8431	100.2051\\
0.8603	100.1836\\
0.8799	100.1510\\
0.9025	100.1041\\
0.9293	100.0375\\
0.9632	99.9405\\
1.0213	99.7580\\
1.0754	99.5937\\
1.1093	99.5044\\
1.1372	99.4428\\
1.1613	99.3999\\
1.1826	99.3708\\
1.2018	99.3519\\
1.2194	99.3410\\
1.2359	99.3364\\
1.2520	99.3369\\
1.2684	99.3426\\
1.2858	99.3541\\
1.3048	99.3726\\
1.3261	99.4003\\
1.3505	99.4400\\
1.3795	99.4964\\
1.4171	99.5800\\
1.5411	99.8640\\
1.5718	99.9192\\
1.5981	99.9577\\
1.6214	99.9842\\
1.6425	100.0014\\
1.6621	100.0116\\
1.6807	100.0159\\
1.6990	100.0152\\
1.7177	100.0095\\
1.7375	99.9982\\
1.7591	99.9802\\
1.7833	99.9536\\
1.8113	99.9156\\
1.8459	99.8605\\
1.8996	99.7652\\
1.9644	99.6523\\
2.0004	99.5984\\
2.0301	99.5616\\
2.0563	99.5361\\
2.0800	99.5193\\
2.1021	99.5091\\
2.1233	99.5045\\
2.1442	99.5049\\
2.1656	99.5102\\
2.1881	99.5208\\
2.2126	99.5378\\
2.2401	99.5628\\
2.2726	99.5988\\
2.3157	99.6540\\
2.4228	99.7941\\
2.4579	99.8311\\
2.4879	99.8565\\
2.5150	99.8735\\
2.5402	99.8840\\
2.5644	99.8890\\
2.5883	99.8891\\
2.6127	99.8845\\
2.6384	99.8746\\
2.6664	99.8587\\
2.6982	99.8350\\
2.7372	99.7999\\
2.8036	99.7328\\
2.8625	99.6759\\
2.9011	99.6447\\
2.9339	99.6238\\
2.9637	99.6101\\
2.9918	99.6023\\
3.0192	99.5994\\
3.0467	99.6013\\
3.0753	99.6081\\
3.1061	99.6202\\
3.1408	99.6392\\
3.1833	99.6679\\
3.2680	99.7323\\
3.3210	99.7692\\
3.3606	99.7914\\
3.3954	99.8058\\
3.4278	99.8143\\
3.4592	99.8178\\
3.4907	99.8166\\
3.5234	99.8106\\
3.5589	99.7993\\
3.5997	99.7812\\
3.6540	99.7518\\
3.7645	99.6907\\
3.8086	99.6722\\
3.8474	99.6608\\
3.8839	99.6548\\
3.9198	99.6535\\
3.9566	99.6569\\
3.9960	99.6653\\
4.0409	99.6797\\
4.1012	99.7042\\
4.2139	99.7508\\
4.2625	99.7654\\
4.3057	99.7738\\
4.3470	99.7771\\
4.3886	99.7759\\
4.4326	99.7699\\
4.4826	99.7583\\
4.5507	99.7375\\
4.6640	99.7026\\
4.7181	99.6910\\
4.7666	99.6853\\
4.8139	99.6842\\
4.8629	99.6877\\
4.9178	99.6964\\
4.9911	99.7129\\
5.1193	99.7422\\
5.1791	99.7508\\
5.2338	99.7541\\
5.2888	99.7528\\
5.3489	99.7468\\
5.4266	99.7342\\
5.5790	99.7085\\
5.6451	99.7026\\
5.7076	99.7015\\
5.7734	99.7050\\
5.8543	99.7141\\
6.0488	99.7379\\
6.1220	99.7411\\
6.1951	99.7397\\
6.2806	99.7334\\
6.5265	99.7122\\
6.6093	99.7115\\
6.7001	99.7154\\
7.0221	99.7337\\
7.1213	99.7317\\
7.2766	99.7231\\
7.4155	99.7173\\
7.5255	99.7173\\
7.6632	99.7222\\
7.8693	99.7293\\
7.9970	99.7287\\
8.2189	99.7219\\
8.3740	99.7200\\
8.5393	99.7227\\
8.8062	99.7272\\
8.9952	99.7249\\
9.2738	99.7218\\
9.5302	99.7247\\
9.7712	99.7256\\
10.0000	99.7234\\
};
\addplot [color=mycolor2, forget plot]
  table[row sep=crcr]{%
0.0000	0.0000\\
0.0203	30.7321\\
0.0406	55.1288\\
0.0609	74.4875\\
0.0811	89.7733\\
0.1013	101.9034\\
0.1215	111.5235\\
0.1417	119.1485\\
0.1618	125.1624\\
0.1819	129.9299\\
0.2020	133.7087\\
0.2221	136.7042\\
0.2423	139.0908\\
0.2625	140.9845\\
0.2829	142.5039\\
0.3035	143.7251\\
0.3244	144.7142\\
0.3457	145.5228\\
0.3676	146.1942\\
0.3904	146.7634\\
0.4145	147.2588\\
0.4405	147.7047\\
0.4689	148.1175\\
0.5004	148.5122\\
0.5350	148.8914\\
0.5710	149.2389\\
0.6057	149.5330\\
0.6374	149.7658\\
0.6658	149.9434\\
0.6912	150.0758\\
0.7139	150.1719\\
0.7342	150.2397\\
0.7522	150.2851\\
0.7681	150.3138\\
0.7822	150.3303\\
0.7947	150.3380\\
0.8064	150.3395\\
0.8180	150.3355\\
0.8304	150.3255\\
0.8443	150.3074\\
0.8601	150.2785\\
0.8783	150.2349\\
0.8994	150.1719\\
0.9244	150.0823\\
0.9554	149.9535\\
1.0004	149.7453\\
1.0756	149.3978\\
1.1092	149.2640\\
1.1367	149.1717\\
1.1603	149.1073\\
1.1810	149.0631\\
1.1992	149.0342\\
1.2154	149.0167\\
1.2300	149.0075\\
1.2436	149.0046\\
1.2570	149.0070\\
1.2709	149.0149\\
1.2860	149.0294\\
1.3029	149.0528\\
1.3220	149.0877\\
1.3440	149.1379\\
1.3700	149.2092\\
1.4022	149.3113\\
1.4505	149.4813\\
1.5195	149.7222\\
1.5538	149.8258\\
1.5818	149.8970\\
1.6059	149.9468\\
1.6272	149.9810\\
1.6462	150.0032\\
1.6634	150.0165\\
1.6793	150.0229\\
1.6946	150.0237\\
1.7099	150.0195\\
1.7260	150.0097\\
1.7436	149.9931\\
1.7632	149.9676\\
1.7855	149.9306\\
1.8116	149.8779\\
1.8436	149.8026\\
1.8899	149.6808\\
1.9665	149.4793\\
2.0011	149.4012\\
2.0295	149.3477\\
2.0541	149.3104\\
2.0760	149.2851\\
2.0959	149.2689\\
2.1143	149.2598\\
2.1318	149.2564\\
2.1491	149.2580\\
2.1670	149.2648\\
2.1860	149.2774\\
2.2068	149.2972\\
2.2302	149.3262\\
2.2574	149.3676\\
2.2909	149.4274\\
2.3409	149.5270\\
2.4112	149.6659\\
2.4468	149.7263\\
2.4761	149.7676\\
2.5017	149.7962\\
2.5248	149.8155\\
2.5461	149.8273\\
2.5664	149.8333\\
2.5862	149.8341\\
2.6062	149.8301\\
2.6271	149.8207\\
2.6497	149.8053\\
2.6748	149.7821\\
2.7038	149.7488\\
2.7395	149.7003\\
2.7968	149.6137\\
2.8585	149.5228\\
2.8950	149.4768\\
2.9254	149.4454\\
2.9524	149.4238\\
2.9772	149.4098\\
3.0005	149.4019\\
3.0232	149.3992\\
3.0459	149.4013\\
3.0694	149.4084\\
3.0945	149.4210\\
3.1222	149.4405\\
3.1541	149.4688\\
3.1944	149.5111\\
3.3329	149.6623\\
3.3665	149.6894\\
3.3962	149.7077\\
3.4237	149.7194\\
3.4499	149.7255\\
3.4757	149.7267\\
3.5019	149.7232\\
3.5294	149.7146\\
3.5592	149.7003\\
3.5931	149.6786\\
3.6352	149.6459\\
3.7858	149.5234\\
3.8214	149.5032\\
3.8535	149.4901\\
3.8838	149.4827\\
3.9134	149.4802\\
3.9433	149.4824\\
3.9745	149.4895\\
4.0084	149.5020\\
4.0474	149.5216\\
4.0988	149.5530\\
4.2120	149.6237\\
4.2542	149.6438\\
4.2912	149.6565\\
4.3258	149.6636\\
4.3597	149.6658\\
4.3940	149.6634\\
4.4302	149.6562\\
4.4704	149.6433\\
4.5195	149.6225\\
4.6900	149.5455\\
4.7326	149.5338\\
4.7722	149.5276\\
4.8110	149.5261\\
4.8508	149.5293\\
4.8936	149.5374\\
4.9435	149.5517\\
5.0183	149.5784\\
5.1059	149.6087\\
5.1581	149.6219\\
5.2044	149.6290\\
5.2490	149.6313\\
5.2944	149.6290\\
5.3436	149.6218\\
5.4029	149.6084\\
5.5990	149.5597\\
5.6522	149.5536\\
5.7034	149.5522\\
5.7564	149.5553\\
5.8163	149.5636\\
5.9022	149.5805\\
6.0133	149.6018\\
6.0774	149.6093\\
6.1364	149.6117\\
6.1966	149.6096\\
6.2651	149.6025\\
6.3760	149.5857\\
6.4756	149.5724\\
6.5465	149.5675\\
6.6143	149.5673\\
6.6880	149.5717\\
6.7914	149.5829\\
6.9262	149.5968\\
7.0067	149.6004\\
7.0855	149.5994\\
7.1784	149.5936\\
7.4181	149.5760\\
7.5079	149.5755\\
7.6088	149.5796\\
7.8942	149.5942\\
7.9995	149.5930\\
8.1507	149.5862\\
8.3109	149.5803\\
8.4306	149.5806\\
8.5965	149.5861\\
8.7752	149.5906\\
8.9159	149.5895\\
9.3181	149.5830\\
9.8805	149.5869\\
10.0000	149.5851\\
};
\addplot [color=mycolor3, forget plot]
  table[row sep=crcr]{%
0.0000	0.0000\\
0.0203	40.9762\\
0.0406	73.5051\\
0.0609	99.3166\\
0.0811	119.6978\\
0.1013	135.8713\\
0.1215	148.6983\\
0.1417	158.8654\\
0.1618	166.8844\\
0.1819	173.2420\\
0.2020	178.2815\\
0.2221	182.2767\\
0.2423	185.4602\\
0.2626	187.9973\\
0.2830	190.0219\\
0.3036	191.6487\\
0.3244	192.9604\\
0.3457	194.0377\\
0.3676	194.9313\\
0.3903	195.6849\\
0.4143	196.3405\\
0.4401	196.9282\\
0.4684	197.4743\\
0.4999	197.9985\\
0.5347	198.5053\\
0.5710	198.9717\\
0.6058	199.3648\\
0.6376	199.6766\\
0.6660	199.9140\\
0.6914	200.0913\\
0.7141	200.2205\\
0.7344	200.3119\\
0.7524	200.3736\\
0.7682	200.4129\\
0.7819	200.4357\\
0.7939	200.4472\\
0.8045	200.4509\\
0.8147	200.4488\\
0.8255	200.4407\\
0.8375	200.4248\\
0.8513	200.3980\\
0.8672	200.3561\\
0.8856	200.2940\\
0.9071	200.2049\\
0.9328	200.0783\\
0.9651	199.8953\\
1.0161	199.5771\\
1.0767	199.2060\\
1.1104	199.0263\\
1.1378	198.9028\\
1.1613	198.8162\\
1.1818	198.7567\\
1.1998	198.7174\\
1.2155	198.6933\\
1.2294	198.6801\\
1.2419	198.6745\\
1.2537	198.6749\\
1.2657	198.6806\\
1.2787	198.6929\\
1.2932	198.7137\\
1.3097	198.7462\\
1.3286	198.7942\\
1.3506	198.8630\\
1.3767	198.9602\\
1.4094	199.1003\\
1.4610	199.3438\\
1.5223	199.6278\\
1.5563	199.7649\\
1.5840	199.8593\\
1.6077	199.9252\\
1.6285	199.9705\\
1.6469	200.0004\\
1.6632	200.0185\\
1.6779	200.0282\\
1.6916	200.0314\\
1.7049	200.0294\\
1.7188	200.0218\\
1.7338	200.0076\\
1.7505	199.9849\\
1.7695	199.9506\\
1.7913	199.9013\\
1.8170	199.8313\\
1.8487	199.7311\\
1.8951	199.5680\\
1.9689	199.3092\\
2.0031	199.2059\\
2.0310	199.1349\\
2.0550	199.0854\\
2.0762	199.0515\\
2.0951	199.0294\\
2.1122	199.0162\\
2.1280	199.0099\\
2.1432	199.0090\\
2.1585	199.0132\\
2.1746	199.0230\\
2.1922	199.0397\\
2.2118	199.0652\\
2.2341	199.1023\\
2.2601	199.1549\\
2.2920	199.2302\\
2.3381	199.3519\\
2.4153	199.5556\\
2.4498	199.6337\\
2.4782	199.6874\\
2.5028	199.7248\\
2.5247	199.7502\\
2.5446	199.7665\\
2.5630	199.7756\\
2.5805	199.7790\\
2.5978	199.7773\\
2.6156	199.7705\\
2.6346	199.7579\\
2.6554	199.7381\\
2.6788	199.7090\\
2.7060	199.6675\\
2.7394	199.6077\\
2.7892	199.5082\\
2.8599	199.3680\\
2.8955	199.3074\\
2.9248	199.2660\\
2.9504	199.2373\\
2.9735	199.2180\\
2.9948	199.2061\\
3.0150	199.2001\\
3.0348	199.1993\\
3.0548	199.2033\\
3.0757	199.2126\\
3.0982	199.2281\\
3.1232	199.2512\\
3.1521	199.2844\\
3.1877	199.3328\\
3.2443	199.4186\\
3.3069	199.5113\\
3.3435	199.5576\\
3.3739	199.5892\\
3.4009	199.6108\\
3.4256	199.6249\\
3.4489	199.6329\\
3.4716	199.6357\\
3.4943	199.6336\\
3.5177	199.6266\\
3.5427	199.6140\\
3.5703	199.5947\\
3.6021	199.5665\\
3.6421	199.5245\\
3.7830	199.3704\\
3.8164	199.3436\\
3.8460	199.3255\\
3.8734	199.3141\\
3.8996	199.3082\\
3.9254	199.3071\\
3.9516	199.3108\\
3.9791	199.3196\\
4.0090	199.3342\\
4.0430	199.3562\\
4.0855	199.3895\\
4.2321	199.5095\\
4.2679	199.5302\\
4.3001	199.5438\\
4.3304	199.5516\\
4.3600	199.5544\\
4.3898	199.5525\\
4.4208	199.5458\\
4.4543	199.5337\\
4.4927	199.5147\\
4.5424	199.4846\\
4.6652	199.4080\\
4.7066	199.3887\\
4.7433	199.3766\\
4.7778	199.3699\\
4.8116	199.3681\\
4.8460	199.3710\\
4.8824	199.3787\\
4.9230	199.3922\\
4.9733	199.4140\\
5.1312	199.4863\\
5.1746	199.4991\\
5.2145	199.5061\\
5.2533	199.5084\\
5.2927	199.5060\\
5.3348	199.4987\\
5.3830	199.4856\\
5.4494	199.4624\\
5.5559	199.4251\\
5.6078	199.4120\\
5.6540	199.4050\\
5.6985	199.4028\\
5.7439	199.4052\\
5.7931	199.4124\\
5.8525	199.4260\\
6.0463	199.4744\\
6.0995	199.4807\\
6.1506	199.4822\\
6.2034	199.4792\\
6.2629	199.4711\\
6.3467	199.4546\\
6.4620	199.4324\\
6.5260	199.4249\\
6.5850	199.4224\\
6.6451	199.4245\\
6.7134	199.4316\\
6.8233	199.4483\\
6.9236	199.4618\\
6.9944	199.4668\\
7.0621	199.4670\\
7.1355	199.4627\\
7.2376	199.4516\\
7.3748	199.4374\\
7.4551	199.4337\\
7.5338	199.4347\\
7.6264	199.4406\\
7.8674	199.4583\\
7.9570	199.4588\\
8.0578	199.4546\\
8.3423	199.4400\\
8.4474	199.4412\\
8.5972	199.4479\\
8.7587	199.4539\\
8.8781	199.4537\\
9.0421	199.4483\\
9.2228	199.4436\\
9.3631	199.4447\\
9.7692	199.4513\\
10.0000	199.4469\\
};
\end{axis}

\begin{axis}[%
width=7.677in,
height=6.104in,
at={(0in,0in)},
scale only axis,
xmin=0.0000,
xmax=1.0000,
ymin=0.0000,
ymax=1.0000,
axis line style={draw=none},
ticks=none,
axis x line*=bottom,
axis y line*=left
]
\end{axis}
\end{tikzpicture}%
%    \caption{Settling Time $\dot\varphi$}
%\end{figure}

