\pagestyle{milan}
\section{Sensitivitätsanalyse der Modellparameter} \label{sec:sesitivitaetsanalyse}
In diesem Abschnitt wird eine Parameter- und Sensivitätsanalysise durchgeführt. 
Es wird dabei die Auswirkung von der Varianz von bestimmten Modellparametern auf die Varianz der Ausgangsparameter untersucht.

Ziel der Sensitivitätsanalyse ist es, wichtige Parameter zu identifizieren und daraus eine Optimierung der Parameter zu ermitteln.

Das Ergebniss der Sensitivitätsanalyse dient zum weiterne Verständnis des mathematischen Modelles bzw. dem zugrundeliegenden Simulationsmodell.
\subsection{Lokale und globlale Sensitivitätsanalyse}

Die verschiedenen Verfahren zur Sensitivitätsanalyse lassen sich in drei Kategorien einteilen: Lokale, globale Sensitivitätsanalyse und der sogenannenten Screening Methode.

Bei der lokalen Sensitivitätsanalyse wird für bestimmte Werte der Ausgnagsgrößen der Einfluss der Eingangsgrößen untersucht. Dabei wird immer ein Parameter variiert und die restlichen konstant gehalten (One-At-a-Time-Methode, OAT).
Die Sensitivitätsanalyse wird so für jeden Parameter einzeln durchgeführt und abschließend kann die spezifische Sensivität der einzelnen Parameter ermittelt werden.
Mathematisch entspricht dies den partiellen Ableitungen der Parameter bezüglich der Ausgangsgrößen

\subsubsection*{One-factor-at-a-time ($\pm \SI{20}{\percent}, \pm 1\, \sigma$)}

\begin{equation}
    sensitivity=\frac{\Delta Y}{\Delta X_{i}} \quad \textrm{Für jeden Parameter}\,X_i, i=1,\dots,n
\end{equation}
\begin{itemize}
    \item Nur lokale Variatizion um Arbeitspunk 
    \item Keine Korrelation zwischen Parametern
    \item Standartabweichung benötigt Annahme zur Distribution und überspannt nicht den gesammten Wertebereich
\end{itemize}

\subsubsection*{Ausdruck als Partial-Ableitung}
\begin{equation}
    sensitivity=\frac{\partial Y}{\partial X_i}
\end{equation}